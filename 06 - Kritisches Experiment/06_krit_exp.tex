\documentclass[12pt,german]{article}

\usepackage[left=2cm, right=2cm, top=2cm, bottom=3.5cm, landscape=false]{geometry}

\usepackage{graphicx}
\usepackage{float}

\usepackage{tabularx}
\newcolumntype{R}{>{\raggedleft\arraybackslash}X}
\newcolumntype{L}{>{\raggedright\arraybackslash}X}
\newcolumntype{C}{>{\centering\arraybackslash}X}
\usepackage{booktabs}
\usepackage{dcolumn}
\usepackage{multirow}

\usepackage[ngerman]{babel}

\usepackage{amsmath}

\title{\vspace{-1.5cm}Protokoll Kritisches Experiment}
\author{Fuchs, Gutmann, Kosbab, Kowal, Steindorf, Fälker, Richter}
\setlength\parindent{0pt}


\begin{document}
    \maketitle
    \tableofcontents

    \section{Kurzbeschreibung des Versuches}
    \begin{itemize}
        \item Zu Beginn wird die Funktionsfähigkeit der Sicherheitssysteme getestet, indem manuell eine Totalabschaltung ausgelöst wird.
        \item Die Anfahrneutronenquelle wird eingefahren und das Signal \glqq Kernhälften zusammen\grqq mit dem Schalter \glqq Simulation Kernhälften zusammen\grqq überbrückt, um auch bei getrennten Kernhälften eine Veränderung der Steuerstäbe zu ermöglichen.
        \item Es wird die Untergrundaktivität bei ein- und ausgefahrenen Steuerstäben gemessen.
        \item Anschließend wird die untere Kernhälfte zunächst um 100 Digits angehoben, wonach die Impulsrate mit ein- und ausgefahrenen Steuerstäben gemessen wird.
        \item Aus den Impulsraten wird berechnet, um wie viele Digits die Kernhälfte erneut angehoben werden darf.
        \item Die letzten beiden Schritte werden wiederholt, bis durch Extrapolation der Schnittpunkte mit der X-Achse die Ungleichung \(X_\text{krit, aus}(i) < X_\text{max} < X_\text{krit, ein}(i)\) zuverlässig erfüllt ist.
    \end{itemize}

    \section{Messwerttabellen}

    \begin{table}[H]
        \begin{tabularx}{\textwidth}{L|R|R|R|R|R|R|R|R}
            \toprule
            Hubhöhe & \centering Stab- & \multicolumn{3}{c|}{Zählraten} &  &  &  &  \\
            $[x_i / dgts.]$ & \centering stellung & \centering $N_{i, 1}$ & \centering $N_{i, 2}$ & \centering $\overline{N_i}$ & \centering $N_0 / \overline{N_i}$ & \centering $k_i$ & \centering $M_i$ & \multicolumn{1}{c}{$\rho_i$} \\
            \midrule
            
        \end{tabularx}
    \end{table}

	\begin{table}[H]
		\centering
		\begin{tabularx}{\textwidth}{L|L|R|R|R|R}
			\toprule
			& \textbf{Steuerstabs\newline stellung} & \textbf{Messung 1} & \textbf{Messung 2} & \textbf{Messung 3} & \textbf{Mittelwert \(N_0\)} \\
			\midrule
			\multirow{2}{*}{WB 1} & ein (0) & 7.4 & 7.3 & 7.6 & 7.43 \\
			& aus (4000) & 9.4 & 9.3 & 9.6 & 9.43 \\
			\midrule
			\multirow{2}{*}{WB 2} & ein (0) & 6.9 & 6,6 & 6,7 & 6,73 \\
			& aus (4000) & 10,4 & 10,3 & 10,1 & 10,26 \\
			\bottomrule
		\end{tabularx}
		\caption{Bestimmung der Ausgangswerte \(N_0\)}
	\end{table}
\end{document}