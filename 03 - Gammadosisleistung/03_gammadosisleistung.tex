\documentclass[12pt,german]{article}

\usepackage[left=2cm, right=2cm, top=2cm, bottom=3.5cm, landscape=false]{geometry}

\usepackage{graphicx}
\usepackage{float}

\usepackage{tabularx}
\usepackage{booktabs}
\usepackage{dcolumn}

% \usepackage{pgfplots}
% \pgfplotsset{compat=1.16}

\usepackage[ngerman]{babel}

\usepackage{amsmath}

\title{\vspace{-1.5cm}Protokoll Gammadosisleistung}
\author{Fuchs, Fälker, Gutmann, Kosbab, Kowal, Richter, Steindorf}

\begin{document}
    \maketitle
    \tableofcontents

    \section{Kurzbeschreibung des Versuches}
    \begin{itemize}
        \item Die Messgeräte werden gemäß Praktikumsanleitung in Betrieb genommen. 
        \item Bei geschlossenem Behälter wird die Dosisleistung der Quelle gemessen.
        \item Die Quelle wird im Versuchsaufbau fixiert.
        \item Mit den drei Messgeräten wird nacheinander in verschiedenen Abständen die Dosisleistung gemessen.
        \item Anschließend werden Abschirmmaterialien (Leichtbeton, Schwerbeton und Blei) vor der Quelle angebracht und die Messungen erneut durchgeführt.
    \end{itemize}

    \section{Messwerttabelle ohne Abschirmung}
    \begin{table}[H]
        \begin{tabularx}{\textwidth}{X|X|X|X|X}
            \toprule
            \( \mathbf{Abstand} \) & \( \mathbf{P_{x; LB 133}} \) & \( \mathbf{P_{x; RGD 27091}} \) & \( \mathbf{P_{x; FH40G}} \) & \( \mathbf{\overline{P_x}} \) \\
            $[m]$ & $[\mu Sv/h]$ & $[\mu Sv/h]$ & $[\mu Sv/h]$ & $[\mu Sv/h]$ \\
            \midrule
            $   0.02 $ & $ 4500.00 $ & $ 1580.00 $ & $ 4600.00 $ & $ 3560.00 $ \\
            $   0.07 $ & $ 1600.00 $ & $ 840.00 $ & $ 1150.00 $ & $ 1196.67 $ \\
            $   0.12 $ & $ 750.00 $ & $ 445.00 $ & $ 535.00 $ & $ 576.67 $ \\
            $   0.22 $ & $ 250.00 $ & $ 182.00 $ & $ 185.00 $ & $ 205.67 $ \\
            $   0.32 $ & $ 140.00 $ & $ 100.00 $ & $ 101.00 $ & $ 113.67 $ \\
            $   0.42 $ & $  80.00 $ & $  65.00 $ & $  63.30 $ & $  69.43 $ \\
            $   0.52 $ & $  50.00 $ & $  40.00 $ & $  41.30 $ & $  43.77 $ \\
            $   0.62 $ & $  45.00 $ & $  32.00 $ & $  29.60 $ & $  35.53 $ \\
            $   0.82 $ & $  30.00 $ & $  19.00 $ & $  19.20 $ & $  22.73 $ \\
            $   1.02 $ & $  17.00 $ & $  11.00 $ & $  12.00 $ & $  13.33 $ \\
            $   1.22 $ & $  10.00 $ & $   8.80 $ & $   8.42 $ & $   9.07 $ \\
            $   1.52 $ & $   7.50 $ & $   6.50 $ & $   6.90 $ & $   6.97 $ \\
            $   2.02 $ & $   4.50 $ & $   3.05 $ & $   3.58 $ & $   3.71 $ \\
            $   2.52 $ & $   3.50 $ & $   2.00 $ & $   2.75 $ & $   2.75 $ \\
            $   3.02 $ & $   2.25 $ & $   1.40 $ & $   1.80 $ & $   1.82 $ \\
            \bottomrule
        \end{tabularx}
        \caption{Messwerttabelle ohne Abschirmung}
    \end{table}

    \section{Graphische Darstellung von $P_x$ in Abhängigkeit vom Abstand r}
    \begin{figure}[H]
        \centering
        \includegraphics[width=0.85\textwidth]{03_Dosisleistung.png}
        \caption{$P_x$ in Abhängigkeit vom Abstand zur Quelle}
    \end{figure}

    % a(t) = A0 * e^(-t*ln(2)/T0,5)
    % a(29,75) = A0 * e(- ln(2) * 29,75/ 30)

    \section{Berechnung von $P_x$ aus gegebener Aktivität und Vergleich mit den Messwerten}
    \subsection{Berechnung}
    Zunächst berechnet sich die Aktivität \(A_(t)\) aus:
    \begin{align*}
        T_{1/2} &= 30a \\
        A_0 &= 0,26 \cdot 10^9 Bq \, \text{(Aktivität zum Zeitpunkt 0 - Febr. 1993)} \\
        A(t) &= A_0 \cdot e^{-\lambda \cdot t} \\
        &= A_0 \cdot e^{-t \cdot \frac{\ln(2)}{T_{1/2}}} \\
    \end{align*}
    Damit beträgt die Aktivität zum jetzigen Zeitpunkt (Dezember 2022 - \( t \approx 29,75 a \)):
    \begin{align*}
        A(29,75\, a) &= A_0 \cdot e^{-29,75\, a \cdot \frac{0.693}{30\, a}} \\
                     &= A_0 \cdot e^{-0.6874} \\
                     &= 0.1308\, GBq
    \end{align*}
    $P_x$ berechnet sich im Anschluss durch:
    \begin{align*}
        &K = 0,0925\, mSv·\cdot m^2 / (h \cdot GBq) \\
        &P_x = K \cdot \frac{A}{r^2} \\
    \end{align*}
    wobei $A$ die soeben berechnete Aktivität der Strahlenquelle, $K$ die Dosiskonstante und $r$ den Abstand des Messgeräts von der Quelle beschreibt.
    \subsection{Vergleich}
    \begin{table}[H]
        \centering
        \begin{tabularx}{1\textwidth}{X|X|X|X|X}
            \toprule
            \( \mathbf{Abstand} \) & \( \mathbf{P_{x; LB 133}} \) & \( \mathbf{P_{x; RGD 27091}} \) & \( \mathbf{P_{x; FH40G}} \) & \( \mathbf{P_{x; berechnet}} \) \\
            $[m]$ & $[\mu Sv/h]$ & $[\mu Sv/h]$ & $[\mu Sv/h]$ & $[\mu Sv/h]$ \\
            \midrule
            $   0.02 $ & $ 4500.00 $ & $ 1580.00 $ & $ 4600.00 $ & $ 30236.65 $ \\
            $   0.07 $ & $ 1600.00 $ & $ 840.00 $ & $ 1150.00 $ & $ 2468.30 $ \\
            $   0.12 $ & $ 750.00 $ & $ 445.00 $ & $ 535.00 $ & $ 839.91 $ \\
            $   0.22 $ & $ 250.00 $ & $ 182.00 $ & $ 185.00 $ & $ 249.89 $ \\
            $   0.32 $ & $ 140.00 $ & $ 100.00 $ & $ 101.00 $ & $ 118.11 $ \\
            $   0.42 $ & $  80.00 $ & $  65.00 $ & $  63.30 $ & $  68.56 $ \\
            $   0.52 $ & $  50.00 $ & $  40.00 $ & $  41.30 $ & $  44.73 $ \\
            $   0.62 $ & $  45.00 $ & $  32.00 $ & $  29.60 $ & $  31.46 $ \\
            $   0.82 $ & $  30.00 $ & $  19.00 $ & $  19.20 $ & $  17.99 $ \\
            $   1.02 $ & $  17.00 $ & $  11.00 $ & $  12.00 $ & $  11.63 $ \\
            $   1.22 $ & $  10.00 $ & $   8.80 $ & $   8.42 $ & $   8.13 $ \\
            $   1.52 $ & $   7.50 $ & $   6.50 $ & $   6.90 $ & $   5.23 $ \\
            $   2.02 $ & $   4.50 $ & $   3.05 $ & $   3.58 $ & $   2.96 $ \\
            $   2.52 $ & $   3.50 $ & $   2.00 $ & $   2.75 $ & $   1.90 $ \\
            $   3.02 $ & $   2.25 $ & $   1.40 $ & $   1.80 $ & $   1.33 $ \\
            \bottomrule
        \end{tabularx}
        \caption{Dosisleistungen aus der gegebenen Aktivität der Cs-137-Quelle}
    \end{table}
    
    \section{Eignung der drei Detektoren und Ursachen der Abweichungen}
    Es ist zu sehen, dass für kleine Abstände (und somit höheren Strahlungsdosen) der theoretisch berechnete Wert deutlich über den gemessenen Werten liegt.
    Diese Abweichungen lassen sich unter anderem auf den Aufbau der Messgeräte zurückführen. \\ 
    Je nach Messgerät sind die verwendeten Ionisationskammern unterschiedlich groß und dementsprechend auch für verschiedene Einsatzzwecke konzipiert. \\
    Eine größere Ionisationskammer (z.B. RGD27091) kann baubedingt nicht in einem Abstand von 2 cm zur Quelle messen. \\
    Dafür bietet diese aber den Vorteil Strahlungen mit einer hohen Genauigkeit messen zu können (Anzeigefehler \(\leq\) 5 \%). \\
    Zusätzlich lässt sich hier die Entfernung nur bedingt Einschätzen, da die gemessene Strahlung gemittelt über die gesamte Größe der Ionisationskammer berechnet und angezeigt wird. \\
    Kleinere Ionsiationskammern hingegen bieten den Vorteil sehr nah an der Quelle messen zu können, was jedoch auf Kosten der Genauigkeit geht (FH 40G \& LB 133-1: Anzeigefehler \(\leq\) 15 \%) \\
    Ab 32 cm entsprechen die gemessenen Dosisleistungen annähernd den berechneten Werten, wobei sich der maximale Fehler auf ca. ± 8, 1 \% beläuft. \\
    

    \section{Berechnung der linearen Schwächungskoeffizienten}
    \begin{table}[H]
        \begin{tabularx}{\textwidth}{X|X|X|X|X|X|X|X|X|X}
            \toprule
            \textbf{d} $[m]$ & \multicolumn{3}{c|}{\textbf{\( P_{x; Leichtbeton} \)} $[\mu Sv/h]$} & \multicolumn{3}{c|}{\textbf{\( P_{x; Schwerbeton} \)} $[\mu Sv/h]$} & \multicolumn{3}{c}{\textbf{\( P_{x; Blei} \)} $[\mu Sv/h]$} \\
            \cmidrule{2-10}
            & \textbf{LB 133} & \textbf{RGD 27091} & \textbf{FH 40G} & \textbf{LB 133} & \textbf{RGD 27091} & \textbf{FH 40G} & \textbf{LB 133} & \textbf{RGD 27091} & \textbf{FH 40G} \\
            \midrule
            $   0,07 $ & $     -  $ & $     -  $ & $     -  $ & $     -  $ & $     -  $ & $     -  $ & $   2,40 $ & $   6,00 $ & $   8,00 $ \\
            $   0,12 $ & $     -  $ & $     -  $ & $     -  $ & $     -  $ & $     -  $ & $     -  $ & $   1,20 $ & $   3,39 $ & $   3,50 $ \\
            $   0,22 $ & $     -  $ & $     -  $ & $     -  $ & $     -  $ & $     -  $ & $     -  $ & $   0,30 $ & $   0,75 $ & $   1,40 $ \\
            $   0,32 $ & $  60,00 $ & $  65,80 $ & $  80,00 $ & $   4,35 $ & $   4,84 $ & $   7,50 $ & $   0,20 $ & $   0,69 $ & $   0,70 $ \\
            $   0,42 $ & $  36,00 $ & $  37,70 $ & $  45,00 $ & $   3,00 $ & $   3,40 $ & $   3,50 $ & $     -  $ & $   0,59 $ & $   0,55 $ \\
            $   0,52 $ & $  24,00 $ & $  26,00 $ & $  35,00 $ & $   2,50 $ & $   2,74 $ & $   2,50 $ & $     -  $ & $   0,43 $ & $   0,30 $ \\
            $   0,62 $ & $  17,50 $ & $  18,80 $ & $  20,00 $ & $   1,00 $ & $   1,85 $ & $   1,80 $ & $     -  $ & $     -  $ & $     -  $ \\
            $   1,02 $ & $   6,40 $ & $   7,38 $ & $   8,00 $ & $   0,50 $ & $   1,00 $ & $   1,00 $ & $     -  $ & $   0,36 $ & $   0,20 $ \\
            $   1,52 $ & $   3,50 $ & $   3,70 $ & $   3,50 $ & $   0,30 $ & $   0,35 $ & $   0,55 $ & $     -  $ & $     -  $ & $     -  $ \\
            \bottomrule
        \end{tabularx}
        \caption{Messwerttabelle mit Abschirmung}
    \end{table}

    Die linearen Schwächungskoeffizienten $\mu$ lassen sich folgendermaßen berechnen:
    \begin{align*}
        P_x &= B_D \cdot P_{x0} \cdot e^{-\mu x} \\[5pt]
        P_x &= P_{x0} \cdot e^{-\mu x} \\[5pt]
        \frac{P_x}{P_{x0}} &= e^{-\mu x} \\[6pt]
        ln\left(\frac{P_x}{P_{x0}}\right) &= -\mu x \\
        \mu &= -\frac{\ln\left(\frac{P_x}{P_{x0}}\right)}{x}
    \end{align*}
    wobei $P_x$ die Dosisleistung mit Abschwächung, $P_{x0}$ die Dosisleistung ohne Abschwächung und $x$ die Dicke der Abschirmung bezeichnet. \\
    In diesem Versuch kann der Dosiszuwachsfaktor $B_D$ vernachlässigt werden, da er durch die Steuung der Comptenquanten annähernd $1$ beträgt.
    \begin{table}[H]
        \centering
        \begin{tabularx}{\textwidth}{X|X|X|X|X}
            \toprule
            \textbf{Abstand} & $ \mu_{\text{RDG}}\, [1/m] $ & $ \mu_{\text{FH40}}\, [1/m]  $ & $ \mu_{\text{LB133}}\, [1/m]  $ & $ \overline{\mu}\, [1/m]  $ \\
            \midrule
            $   0,32 $ & $   2,55 $ & $   2,14 $ & $   2,80 $ & $   2,50 $ \\
            $   0,42 $ & $   2,95 $ & $   2,59 $ & $   2,88 $ & $   2,81 $ \\
            $   0,52 $ & $   2,55 $ & $   2,31 $ & $   1,78 $ & $   2,22 $ \\
            $   0,62 $ & $   3,02 $ & $   2,27 $ & $   4,05 $ & $   3,11 $ \\
            $   1,02 $ & $   2,71 $ & $   2,43 $ & $   3,77 $ & $   2,97 $ \\
            $   1,52 $ & $   3,10 $ & $   3,12 $ & $   3,81 $ & $   3,34 $ \\
            \midrule
            $\emptyset$ & $   2,81 $ & $   2,48 $ & $   3,18 $ & $   2,82 $ \\
            \bottomrule
        \end{tabularx}
        \caption{Lineare Schwächungskoeffizienten für Leichtbeton}
    \end{table}
    \begin{table}[H]
        \centering
        \begin{tabularx}{\textwidth}{X|X|X|X|X}
            \toprule
            \textbf{Abstand} & $ \mu_{\text{RDG}}\, [1/m] $ & $ \mu_{\text{FH40}}\, [1/m]  $ & $ \mu_{\text{LB133}}\, [1/m]  $ & $ \overline{\mu}\, [1/m]  $ \\
            \midrule
            $   0,32 $ & $  15,67 $ & $  15,19 $ & $  14,63 $ & $  15,17 $ \\
            $   0,42 $ & $  15,38 $ & $  14,62 $ & $  15,65 $ & $  15,22 $ \\
            $   0,52 $ & $  13,86 $ & $  13,56 $ & $  14,98 $ & $  14,14 $ \\
            $   0,62 $ & $  17,33 $ & $  13,86 $ & $  16,09 $ & $  15,76 $ \\
            $   1,02 $ & $  15,46 $ & $  12,42 $ & $  14,17 $ & $  14,02 $ \\
            $   1,52 $ & $  15,38 $ & $  14,91 $ & $  13,06 $ & $  14,45 $ \\
            \midrule
            $\emptyset$ & $  15,51 $ & $  14,10 $ & $  14,76 $ & $  14,79 $ \\
            \bottomrule
        \end{tabularx}
        \caption{Lineare Schwächungskoeffizienten für Schwerbeton}
    \end{table}
    \begin{table}[H]
        \centering
        \begin{tabularx}{\textwidth}{X|X|X|X|X}
            \toprule
            \textbf{Abstand} & $ \mu_{\text{RDG}}\, [1/m] $ & $ \mu_{\text{FH40}}\, [1/m]  $ & $ \mu_{\text{LB133}}\, [1/m]  $ & $ \overline{\mu}\, [1/m]  $ \\
            \midrule
            $   0,07 $ & $ 117,16 $ & $ 105,12 $ & $ 105,97 $ & $ 109,41 $ \\
            $   0,12 $ & $ 118,32 $ & $ 101,23 $ & $ 107,35 $ & $ 108,96 $ \\
            $   0,22 $ & $ 128,16 $ & $ 110,16 $ & $ 103,70 $ & $ 114,01 $ \\
            $   0,32 $ & $ 124,29 $ & $  99,72 $ & $ 105,97 $ & $ 109,99 $ \\
            $   0,42 $ & $     -  $ & $  93,51 $ & $  99,60 $ & $  96,56 $ \\
            $   0,52 $ & $     -  $ & $  91,30 $ & $ 102,32 $ & $  96,81 $ \\
            $   1,02 $ & $     -  $ & $  70,13 $ & $  88,85 $ & $  79,49 $ \\
            \midrule
            $\emptyset$ & $ 121,98 $ & $  95,88 $ & $ 101,96 $ & $ 101,75 $ \\
            \bottomrule
        \end{tabularx}
        \caption{Lineare Schwächungskoeffizienten für Blei}
    \end{table}

    \section{Halbwertsdicke und Massenschwächungskoeffizient}
    Die Halbwertsdicke $ x_{1/2} $ sowie die Massenschwächungskoeffizienten $\mu'$ berechnet sich durch:
    \begin{align*}
        &x_{1/2} = \frac{\ln(2)}{\mu} \\
        &\mu' = \frac{\mu}{\rho} \\
    \end{align*}
    wobei $\mu$ den linearen Schwächungskoeffizienten sowie $\rho$ die Dichte des jeweiligen Materials bezeichnet. \\
    \begin{table}[H]
        \centering
        \begin{tabularx}{\textwidth}{X|X|X}
            \toprule
            \textbf{Material} & \( \mathbf{x_{1/2}} \, [cm]\) & \( \mathbf{\mu'} \, [{cm}^2/g] \) \\
            \midrule
            $ Leichbeton $ & $ 24,5 $ & $ 0,047 $ \\
            $ Schwerbeton $ & $ 4,6 $ & $ 0,059 $ \\
            $ Blei $ & $ 0,68 $ & $ 0,087 $ \\
            \bottomrule
        \end{tabularx}
        \caption{Halbwertsdicken und Massenschwächungskoeffizienten}
    \end{table}
    
    \newpage
    \section{Berechnung der erforderlichen Schutzschichtdicke}
    Die erforderlichen Schutzschichtdicke berechnet sich aus der Gammadosisleistung der Quelle \(P_0\), der zu erreichenden Dosisleistung \(P\), 
    der linearen Schwächungskoeffizienten von Luft \(\mu_{\text{Luft}} = 0,01\, \frac{1}{m}\) und Blei \(\mu_{\text{Blei}} =101,75\, \frac{1}{m}\) sowie dem Gesamtabstand \(r\) zur Quelle:
    
    \begin{align*}
        P_0 &= K \cdot \frac{A}{r^2} = 0,0925 \cdot \frac{0,1308}{{0,5}^2} \\
            &= 48,39\, \mu SV/h \\
        \\
        P &= 25\, \mu SV/h \\
        d_{\text{Luft}} + d_{\text{Blei}} &= 0,5m \\
        \\
    % \end{align*}
    % \begin{align*}
        P &= P_0 \cdot e^{-\mu_{\text{Luft}}\, \cdot\, d_{\text{Luft}}} \cdot e^{-\mu_{\text{Blei}}\, \cdot\, d_{\text{Blei}}} \\
        \frac{P}{P_0} &= e^{-\mu_{\text{Luft}}\, \cdot\, d_{\text{Luft}}} \cdot e^{-\mu_{\text{Blei}}\, \cdot\, d_{\text{Blei}}} \\
        \frac{P}{P_0} &= e^{-\mu_{\text{Luft}}\, \cdot\, d_{\text{Luft}}\, -\, \mu_{\text{Blei}}\, \cdot\, d_{\text{Blei}}} \\
        \ln\left(\frac{P}{P_0}\right) &= -\mu_{\text{Luft}}\, \cdot\, d_{\text{Luft}}\, -\, \mu_{\text{Blei}}\, \cdot\, d_{\text{Blei}} \\
        \ln\left(\frac{P}{P_0}\right) &= -\mu_{\text{Luft}}\, \cdot\, (0,5\, m - d_{\text{Blei}})\, -\, \mu_{\text{Blei}}\, \cdot\, d_{\text{Blei}} \\
        \ln\left(\frac{P}{P_0}\right) &= -\mu_{\text{Luft}}\, \cdot\, 0,5\, m\, + \mu_{\text{Luft}}\, \cdot\, d_{\text{Blei}}\, -\, \mu_{\text{Blei}}\, \cdot\, d_{\text{Blei}} \\
        \ln\left(\frac{P}{P_0}\right) + \mu_{\text{Luft}}\, \cdot\, 0,5\, m &= \mu_{\text{Luft}}\, \cdot\, d_{\text{Blei}}\, -\, \mu_{\text{Blei}}\, \cdot\, d_{\text{Blei}} \\
        \ln\left(\frac{P}{P_0}\right) + \mu_{\text{Luft}}\, \cdot\, 0,5\, m &= (\mu_{\text{Luft}}\, -\, \mu_{\text{Blei}})\, \cdot\, d_{\text{Blei}} \\
        d_{\text{Blei}} &= \frac{\ln\left(\frac{P}{P_0}\right) + \mu_{\text{Luft}}\, \cdot\, 0,5\, m}{\mu_{\text{Luft}}\, -\, \mu_{\text{Blei}}} \\
        \\
        d_{\text{Blei}} &= \frac{\ln\left(\frac{25\, \mu SV/h}{48,39\, \mu SV/h}\right) + 0,01\, \frac{1}{m}\, \cdot\, 0,5\, m}{0,01\, \frac{1}{m}\, -\, 101,75\, \frac{1}{m}} \\
        d_{\text{Blei}} &= 0,64\, cm \\
    \end{align*}
    Die erforderlichen Schutzschichtdicke beträgt damit ca. 0,64 cm.

    \section{Einfluss des Abschirmmaterials auf die Strahlenschutzeinrichtungen}
    Die Auswahl des Abschirmmaterials beim Aufbau der Strahlenschutzeinrichtung hängt von vielen unterschiedlichen Faktoren ab:
    \begin{itemize}
        \item die gewünschte Stärke der Strahlungsabsorbtion
        \item die zu absorbierenden Strahlungsarten
        \item der allgemeinen Materialverfügbarkeit
        \item die Anschaffungskosten
        \item bauliche Rahmenbedingungen
        \item eventuellen gesundheitlichen Auswirkungen
    \end{itemize}

\end{document}