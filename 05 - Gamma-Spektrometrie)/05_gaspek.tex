\documentclass[12pt,german]{article}

\usepackage[left=2cm, right=2cm, top=2cm, bottom=3.5cm, landscape=false]{geometry}

\usepackage{graphicx}
\usepackage{float}

\usepackage{tabularx}
\newcolumntype{R}{>{\raggedleft\arraybackslash}X}
\newcolumntype{L}{>{\raggedright\arraybackslash}X}
\newcolumntype{C}{>{\centering\arraybackslash}X}
\usepackage{booktabs}
\usepackage{dcolumn}

\usepackage[ngerman]{babel}

\usepackage{amsmath}

\title{\vspace{-1.5cm}Protokoll Gammadosisleistung}
\author{Fuchs, Gutmann, Kosbab, Kowal, Steindorf, Fälker, Richter}

\begin{document}
    \maketitle
    \tableofcontents

    \section{Kurzbeschreibung des Versuches}
    \begin{itemize}
        \item Mithilfe von Cobalt-60 werden die Detektoren kalibriert, die Kalibrierung wird mit Cäsium-137 validiert.
        \item Auf Millimeterpapier wird die Linearität der Zuordnung von Kanallage zu Energie nachgewiesen.
        \item Es wird eine Kupfer-Probe für 10 Minuten aktiviert, anschließend mit dem Detektor ausgewertet und 10 Minuten später erneut ausgewertet.
        \item Im Reaktor wird die unbekannte Probe für 10 Minuten aktiviert.
        \item Die unbekannte Probe wird anhand der Fotopeaks ausgewertet, aufgrund verschobener Kalibrierung wird sie jedoch gegen eine neue unbekannte Probe ausgetauscht.
        \item Die neue Probe wird anhand der Fotopeaks identifiziert.
    \end{itemize}

    \section{Auswertung von Kupfer}
    

    \section{Messung der ersten unbekannten Probe}
    
\end{document}